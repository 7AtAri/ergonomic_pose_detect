(by Polina Kozyr)

\section{Marketing methods}

Our company needs advertising to make people aware of us. There are several ways to achieve recognition and sales for our company:

\begin{itemize}
    \item Digital Marketing

    Social Media Marketing: Leverage platforms like Instagram, Facebook, TikTok and Twitter to showcase the benefits of the app. We can use engaging content, such as videos and testimonials, to build brand awareness.
    
    Search Engine Optimization (SEO): Optimize the website and app store listings to rank higher in search engine results, making it easier for potential users to discover our app.
    \item Content Marketing

    Blog Posts and Articles: creation of informative blog posts or articles related to posture, health, and well-being. Sharing of practical tips and information that can attract target audience.
    
    Video Content: Development of instructional videos, product demonstrations, or educational content about the importance of maintaining good posture. Videos can be shared on social media, YouTube, and our website.
    \item Influencer Marketing
    
    We can enter into a partnership with influencers in the health and fitness space to promote our posture-controlling app. 
    \item Email Marketing
    
    Build an email list and send newsletters with valuable content, updates, and promotional offers. Email marketing is a direct way to engage with our audience and encourage app usage. But emails will only be used if people have given us permission to send them.
    \item Partnerships and Collaborations

    Formation of partnerships with health and wellness brands, fitness influencers, or ergonomic product manufacturers. Collaborations can enhance app's credibility and broaden its reach.
    \item App Store Optimization (ASO)
    
    Optimization of the app store listings with relevant keywords, compelling descriptions, and high-quality visuals. It is needed for increasing visibility on app stores and attracting downloads.
    \item Free Trials and Promotions

    Free trials to encourage users to try our app. Positive user experiences during trial periods can lead to increased subscriptions or purchases.
    \item Community Engagement

    Creation of online communities, forums, or social media groups where users can discuss posture-related topics, share experiences, and ask questions. Engaging with our community builds brand loyalty. We could make marathons like “21 day of straight posture”.
    \item Events and Sponsorships

    Attendance of wellness events, conferences, or trade shows. Participating in relevant events can provide exposure and networking opportunities.
    \item Leverage User Reviews

    Encouragement of satisfied users to leave positive reviews on app stores or provide testimonials for our marketing materials. 
\end{itemize}

\section{4Ps of marketing}

The 4Ps of marketing are a set of key elements that are considered essential in the development and execution of a marketing strategy \cite{4PsofMarketing}. This analysis is presented in table 6.1.

\begin{table}[H]
    \centering
    \small
    \begin{tabular}{|c|c|}
        \hline
        \parbox{8cm}{\vspace{5pt}
        \textbf{PRODUCT}\\
        •	Intangible product - app\\
        •	Features: real-time posture monitoring, personalized feedback, posture reminder, user-friendly interface\\
        •	Our USP (unique selling point)\\
            “You don’t need to worry about your posture while focusing, we will evaluate and report on your posture for you!“\\
            "We will help you to develop a habit.“\\
            “Don’t interrupt your flow!”\\
        •	Needs: posture correction and constantly remembering about posture\\
        •	Marathons\\
        •	Healthy posture "club"\\
        •	Subscription\\
        •	Family subscription\\
        •	Business subscription models\\
        \vspace{5pt}} & \parbox{7cm}{\vspace{5pt}
        \textbf{PROMOTION}\\
        •	Create Instagram, Facebook and TikTok, WeChat, LinkedIn profiles with content that will be constantly updated\\
        •	Bloggers and influencers can promote the app for commission\\
        •	Paid advertising channels, such as Google Ads, Facebook Ads\\
        •	Family subscription that will have lower price per person than single-person subscription\\
        •	Price for subscription for six or twelve months at once will be cheaper\\
        •	Black Friday\\
        •	Cooperation with health insurances, physiotherapists   \\     
        \vspace{5pt}} \\
    \hline
    \parbox{8cm}{\vspace{5pt}
        \textbf{PRICE}\\
        •	Free trial period, e.g. 7 days\\
        •	Payment for subscription once in a month, six months, year\\
        •	Show the value: If you take care about posture now, you don't have to pay for medicine in the future\\
        •	Custom features for extra charge\\
        •	Multiple people subscription: family subscription\\  
        \vspace{5pt}} & \parbox{7cm}{\vspace{5pt}
        \textbf{PLACE}\\
        •	The app will be available on our website, App store and Google Play     
        \vspace{5pt}} \\
    \hline
    \end{tabular}
    \caption{4Ps of marketing}
\end{table}

\section{Analysis of customer groups}

Our customer base can be segmented based on various factors, including demographics, occupation, family status and income. Different groups of customers are presented in table 6.2.

\begin{table}[H]
    \centering
    \small 
    \begin{tabular}{|p{1.9cm}|p{2cm}|p{2cm}|p{2cm}|p{2.1cm}|p{2cm}|p{2cm}|}
        \hline
        \textbf{Criteria} & \textbf{Segment 1} & \textbf{Segment 2} & \textbf{Segment 3} & \textbf{Segment 4} & \textbf{Segment 5} & \textbf{Segment 6} \\
        \hline
        \textbf{Occupation} & Graduated, Employed & Graduated, Employed	& Students, Graduated or apprenticeship	& Apprenticeship, Students	& not employed, not students & E-Sports, Gamers \\
        \hline
        \textbf{Age} & 50-65 & 30-50 & 25-40 & 20-30 & 20-65 & 15-50 \\
        \hline
        \textbf{Family status} & Married, "Empty nest" & Married, with kids & Singles, married, with kids & Single & Married, Single, with kids & *\\
        \hline
        \textbf{Income} & 60.000-200.000 & 60.000-100.000 & 30.000-80.000 & 10.000-50.000 & * & 0-200.000\\
        \hline
    \end{tabular}
    \caption{Analysis of customer groups}
    \label{tab:example}
\end{table}

We use the “*” sign to denote irrelevant table cells.

\textbf{Group A}: Clients in this group (Segments 2, 3, and 6) are crucial for our company's success and revenue. Segment 3 comprises students, graduate students, and working individuals who spend significant time on computers, potentially experiencing back issues due to a sedentary lifestyle. Segment 2 shares similar characteristics but includes individuals which are employed and have families, possibly interested in a family subscription. Lastly, Segment 6 consists of professional competitive sports players who, while spending considerable time on computers, may benefit from posture tracking. All these segments have a common need to control their posture while working on the computer and have a regular income.

\textbf{Group B}: Customers in this group (Segments 1 and 4) are less likely to be initially interested in our product, but we aim to attract them. Segment 1 consists of working individuals, likely married, possibly less engaged in social media, and preferring traditional health care methods. Developing strategies to appeal to this segment is a future focus. Segment 4 includes students or trainees who are single, not currently concerned with posture correction. Future efforts may involve using bloggers and youth social networks to create engaging content emphasizing the importance of posture control for young individuals.

\textbf{Group C}: Clients in this group are unlikely to find our product relevant. This category consists of individuals who neither work nor study extensively on computers, indicating a limited need for our posture-controlling app.





