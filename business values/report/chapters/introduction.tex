(by Nourhan Omar)\\

\p
In today's digital age, the widespread use of computers has become an integral part of daily life for many people, whether for work, studying, or playing video games. However, given the convenience and mobility offered by laptops, the importance of maintaining good posture often gets overlooked, resulting in a lot of health issues such as back pain and frequent visits to physiotherapists.

\p
Despite efforts by some companies to address this issue by providing ergonomic furniture and standing desks, employees still face challenges in maintaining optimal posture throughout their workday. The shift to remote or hybrid work arrangements following the COVID-19 pandemic has only increased this problem, as many individuals now rely on home office setups that probably lack ergonomic design.

\p
Moreover, the high cost of ergonomic furniture can pose a barrier for both employers and individuals, making it impractical. While exercises and stretches can help with posture-related pain, the demands of a busy workday often leave little time for dedicated physical activity.

\p
Considering these challenges, there is a pressing need for innovative solutions that can easily integrate into work environments, promoting healthy posture habits without affecting overall productivity. By addressing the root causes of poor posture and providing practical, accessible solutions, we can empower individuals to prioritize their physical health and overall well-being.