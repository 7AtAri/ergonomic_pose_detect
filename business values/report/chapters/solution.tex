(by Ari Wahl)

The first goal of this project is to develop a software that helps people to work
ergonomically correct. We will develop an application that will be able to detect the user's posture in real-time
and give feedback to the user. The feedback should be given in a way that users
can correct their posture if necessary. The software should be easy to use and should be able to run on 
different platforms and devices. It should be available and easily accessible to businesses and private users. Ideally, the user
can just use the webcam of their device to use the software. The application will be able to run in the background
and give the feedback in a subtle way. We propose a traffic light colored-scheme that will be displayed as a frame
around the users screen. This means that the frame will be green if the user's posture is correct, yellow if the posture is not optimal 
and should be changed soon and red if the posture is bad and should be changed immediately. 
We will also provide an option to display the feedback as sound. There will be no sound if the posture is correct, a
beep from time to time if the posture is not optimal and a continuous sound if the posture is bad.

As data privacy is an important issue with real-time video analysis, we will develop the software in a way that the raw video data
will not leave the user's device. It will be processed locally in a first step on the user's device and only the processed data will be sent to the server.

To further develop our application we will (optionally) also analyse the data over time and give the user feedback about their posture 
in the form of a dashboard. This dashboard will show the user for example how much time they spent in a good, bad or neutral posture
and which is their most used bad posture, etc. It will also show the user how much time they spent sitting in total. 

As a further step, the application will be extended to optionally also give the user advice on how to compensate for long sitting periods.
This will be done by suggesting the user different exercises once in a while after they have been sitting for a long time.  

For further development we see the potential to grow into an ergonomic workout application branch. This means that
we will have an additional model to detect the user's posture and movements during workout and give them feedback on how to do the exercises correctly.

Another possible area of growth will be physiotherapy applications. 
This means that we will have an additional model to detect the user's posture and movements
during physiotherapy exercises and give them feedback on how to do the exercises correctly.

Since these further developments need additional resources and domain knowledge we will focus on the implementation
of the first goal of this project. 