
by Ari Wahl

For our Ergonomic Pose App "PoseFix" we will use YOLO v8 pose as a base model. 
It can run on mobile devices (Android and Os) with 6-7 frames per second \cite{ultralytics2022}, 
which is more than enough for our application as well as on laptops or desktop computers. For data protection and privacy we will send only the keypoints from the pose detection 
the be evaluated online on our classification layer or alternatively run the model as a lightweight application completely on the users devices. 
Either way, this ensures that there is no threat for businesses or private persons as customers to be victims of spy attacs. Just having keypoints 
would only allow for an extremely abstract representation and is therefore a perfect measure to protect the data and privacy of our customers. 
For the adaption of the YOLOv8 pose model for our application, we train a classification layer on basis of the keypoint representation. 
To evaluate, if a pose is ergonomic or not, we collected a dataset, which uses classification levels from the well established RULA (Rapid Upper Limb Assessment) employee assessment 
worksheed \cite{Holzgreve_2022}. 
Additional implementations that exceed the base model will be a dashboard for monitoring the posture over time and show long term improvements 
to the customer. Also we plan to optionally leverage Explainable AI methods to indicate which joint positions are problematic and show
 in which direction an improvement can be achieved most quickly. To establish more trust among our (potential) customers 
 we will also aim to get some certification(s) that prove the health impact of our application, e.g. TÜV. 
For our applications we use the following modules and packages so far: ultralytics YOLO, openCV, Numpy, Pillow, Cocoa, Quartz, objc, PyObjCTools. 


