by Ari Wahl

For our Ergonomic Pose App "PoseFix" we will use YOLO v8 pose as a base model. 
It can run on mobile devices (Android and Os) with 6-7 frames per second \cite{ultralytics2022}, 
which is more than enough for our application as well as on laptops or desktop computers. For data protection and privacy we will send only the keypoints from the pose detection 
the be evaluated online on our classification layer or alternatively run the model as a lightweight application completely on the users devices. 
Either way, this ensures that there is no threat for businesses or private persons as customers to be victims of spy attacs. Just having keypoints 
would only allow for an extremely abstract representation and is therefore a perfect measure to protect the data and privacy of our customers. 
For the adaption of the YOLOv8 pose model for our application, we train a classification layer on basis of the keypoint representation. 
To evaluate, if a pose is ergonomic or not, we collected a dataset, which uses classification levels from the well established RULA (Rapid Upper Limb Assessment) employee assessment 
worksheed \cite{Holzgreve_2022}. 
Additional implementations that exceed the base model will be a dashboard for monitoring the posture over time and show long term improvements 
to the customer. Also we plan to optionally leverage Explainable AI methods to indicate which joint positions are problematic and show
 in which direction an improvement can be achieved most quickly. To establish more trust among our (potential) customers 
 we will also aim to get some certification(s) that prove the health impact of our application, e.g. TÜV. 
For our applications we use the following modules and packages so far: ultralytics YOLO, openCV, Numpy, Pillow, Cocoa, Quartz, objc, PyObjCTools. 


A PEST-analysis was done to evaluate the external factors that might influence our business. Since this includes political, economic, social and technological factors, 
and we focus on the technological factors in this chapter, we decided to include the PEST-analysis in this chapter. And since we are a startup, we also included a 
SWOT-analysis, that was informed by the PEST-analysis. 

\section{PEST-analysis}

\subsection{Political Factors}

Our business model is influenced by several key political factors, including labor laws, data privacy laws, AI-related regulations, and copyright laws.

\textbf{Labor Laws:} The Workplace Ordinance ArbStättV is crucial for our business, outlining employers' responsibilities towards the ergonomic design and maintenance of workplaces. Our model could benefit from stricter ergonomic requirements \autocite{ArbStattV}.

\textbf{Data Privacy:} Compliance with the General Data Protection Regulation GDPR is essential, especially for products dealing with employee data \autocite{GDPR}.

\textbf{AI-Related Laws:} The forthcoming EU AI Act may impact our use of third-party AI software, affecting costs and accessibility \autocite{EUAIACT}.

\textbf{Copyright Laws:} Ensuring that all third-party software or pretrained models used are either copyright-free or appropriately licensed is crucial \autocite{CopyrightLaws}.

\textbf{Opportunities and Risks:} Current labor laws create a favorable environment for our B2B model, but changes in ergonomic regulations could present risks. The GDPR, while imposing higher initial compliance costs, offers an opportunity to appeal to privacy-conscious customers. The evolving landscape of AI and copyright laws presents both challenges and opportunities.


\subsection{Economic factors}

Right now, one of the economic factors that would influence our business model most is businesses that pressure their employees into returning to the offices again. Less remote work would influence our business model, since the B2C part highly relies on the people working remotely but still wanting to have ergonomic working conditions. It might also deminish the amount of possible buyers in total, because a person working remotely often still has another workplace in the office, doubeling the numbers of workplaces per person. People returning to their offices for all days in the week would in total mean less demand for our product and our profitability would suffer. Another factor in the economy would be the distribution of jobs in general. If the (national) economy would develop into less office / desk jobs overall and more manual labor, this would be a problem for our business model in the beginning. In the long run, the business can also focus on manual labor, where ergonomic posture is also a very strong long term health factor that is often not met. So this could rise the demand for our service from the manual labor side. But it needs a lot more training and training data to include all possible manual labour applications, so this could have a negative impact on our overall profitability. With AI introduced into economy, it is unclear how the job distribution will develop in the next years. Right now AI technologies mostly threaten higher skilled workes in desk jobs, but this not necessarily means more manual labor employment on the other hand. It is vital for our business model to watch these developments closely to take strategic measures at the appropriate time. Also the currently high inflation rates might have a negative impact on the demand for our product since businesses may need to keep investments low to survive. 

\subsection{Social factors}

Social aspects that may impact our business might be changing sensibilities towards AI monitoring. The trend in recent years was people being more and more desensitized for data privacy issues. The wide spread use of social media presence where everyone is giving up at least parts of their privacy voluntarily or just accepting that this is the price for having online attention and connection with friends and acquaintances. This trend is in favour of our product because otherwise people might find it scary (some maybe still do) to be constantly visually monitored at the workplace, even if the data is anonimized at the location before it gets sent to be evaluated by our AI model. A change in lifestyle, social media use and consumer attitudes towards a higher sensibility for this kind of monitoring might therefore threaten our business model. We would need to invest in good marketing, TÜV certifications, etc to assure our customers that their privacy is protected. This could negatively impact our overall profitability. But right now, it seems the trend towards desensibilisation is still unbroken. Another social and cultural aspect impacting our business might be the concept and importance of health in our society. Recent trends lead to a focus on the continuous improvement of healthy behaviour, aiming for longer and better quality of life. Our product greatly profits from this trend, especially in the B2C sector but this also influences the regulations that control the demand in the B2B sector. If this trend changes, the demand of our product might decrease making our business less profitable. For now, the demographics are on our side. A lot of older workes, already suffering from back pain due to unhealthy ergonomics at their workplaces for decades or neglected ergonomics at their remote workplaces that impact older people faster and harder, should stabalize the demand for our product for the next five years. A threat for the demand of our product in the long run though might be the mass retirement of the boomer-generation, that is already happening. The step-by-step retirement of this numerically strong generation might influence the demand of our product. Also younger people that do not want to work five days a week could influence the need and as a result the demand of our product. If people work less and move more in their free time they might suffer less chronic pains or it might take longer to get those even if they do not work under ergonomically good conditions.

\subsection{Technological factors}

Since our product is state of the art in technology and automation the influence of technological factors may, for now, not be this big. It is still necessary to have Research and Development positions to watch out for technological improvements and new technologies, that might lead to better predictions of ergonomic poses or better data privacy prediction, smaller and more ecological AI models and technology, etc. The idea is to stay in a state of the art position regarding our product, so that new developments are not a threat for our business.
The main technological threat to our business model is a big tech company picking up the business idea and incorporate it in their business model. It would then be within an operating system or microsoft 365 applications which would then be ubiquitous and not be a business model for a small or middle sized company anymore. It would drop the demand of our product significantly and abruptly. 

\section{SOWT-analysis}

\begin{figure}[h]
    %\centering
    \includegraphics[width=17.5cm]{SWOT_analysis.png}
\end{figure}

